\documentclass[]{article}
\usepackage{amsmath}
\usepackage{amssymb}
\usepackage{amsfonts}
\usepackage{amsthm}
\usepackage[T1]{fontenc}
\usepackage[utf8]{inputenc}
\usepackage{graphicx}
\usepackage{float}
\usepackage{caption}
\usepackage{subcaption}
\usepackage{booktabs}
\usepackage{siunitx}
\usepackage{url}
\usepackage[brazil]{babel}
\usepackage[fixlanguage]{babelbib}
\usepackage{enumitem}
\usepackage{pgf,tikz}
\usepackage{calc,pgffor,xkeyval}
\usepackage{pgfplots}
\usepackage{dsfont}
\usepackage[bookmarks=true, hidelinks]{hyperref}
\usepackage[a4paper, margin=2cm]{geometry}
\usepackage[toc,page]{appendix}
\usepackage{txfonts}
\usepackage{lmodern}



\pgfplotsset{compat=1.18}

\newcounter{question}[section]
\newenvironment{question}[1][]{\refstepcounter{question}\par\medskip
   \noindent \textbf{Exercício~\thequestion. #1} \rmfamily}{\medskip}


\newcounter{sol}[section]
\newenvironment{sol}[1][]{\refstepcounter{sol}\par\medskip
   \noindent \textbf{Solução~\thesol. #1}  \rmfamily}{\medskip}

\newcounter{theorem}[section]
\newenvironment{theorem}[1][]{\refstepcounter{theorem}\par\medskip
   \noindent \textbf{Teorema~\thetheorem. #1} \rmfamily}{\medskip}

\newcounter{reminder}[section]
\newenvironment{reminder}[1][]{\refstepcounter{reminder}\par\medskip
   \noindent \textbf{Recordação~\thereminder. #1} \rmfamily}{\medskip}



\title{\textbf{MAP5724\\Resolução Numérica de Equações Diferenciais Parciais\\Projeto Elíptico 1}}
\author{Prof.: Antoine Laurain\\Aluno Eric Vargas}
\date{Segundo semestre de 2022.}



\begin{document}

\maketitle
\tableofcontents
\newpage
\section{Parte teórica}
Seja $\Omega = (0,1)^2$ e $\Gamma = \partial \Omega$. Implementar e estudar a convergência para uma discretização de
ordem superior, usando o estêncil compacto de 9 pontos, para a equação de Poisson seguinte
\begin{align}
    -\Delta u &= f \text{ em } \Omega\\
    u &= g \text{ em } \Gamma.
\end{align}

A equação do estêncil compacto é:
\begin{align}
    D_h u &= R_h f,\text{ com }
    D_h =  \frac{h^-2} 6 \left[ 
        \begin{array}{ccc}
            -1& -4& -1\\
            -4& 20& -4\\
            -1& -4& -1
        \end{array}\right], \text { e }
        R_h = \frac 1 {12} \left[ \begin{array}{ccc}            
            &1& \\
            1 &8& 1 \\
            &1&
        \end{array}
        \right]    
\end{align}

\subsection{Consistência}

Para consistência, por Taylor temos:
\begin{align}
    D_h u = -\Delta u  -\frac{h^2} {12} \Delta^2u - \frac{h^4} {360} \left[ 
        \frac {\partial^4}{\partial x^4} 
        +4\frac{\partial^4}{\partial x^2 \partial y^2}
        +\frac{\partial^4}{\partial y^4}
    \right]
    + O(h^6)
\end{align}
De modo similar:
\begin{align*}
    R_h f = f(x,y) + \frac{h^2} {12} \Delta f(x,y) + \frac{h^4} {144} \left[ 
        \frac{ \partial^4f(x,y)}{\partial x^4} +
        \frac{ \partial^4f(x,y)}{\partial y^4}
    \right] + O(h^4),
\end{align*}
como $\Delta f = - \Delta^2 u$ temos que o estêncil $D_h u = R_h f \implies -\Delta u = f + O(h^4)$.

\subsection{Estabilidade}

O estêncil pode ser escrito de forma matricial da forma $L_h u_h = q_h$, onde $L_h$ possui a forma:
\begin{align*}
    L_h = \frac{ h^{-2}   } 6 \left[
        \begin{array}{ccccc}
            T &B& 0 & \dots & 0 \\
            B& T & B & \ddots & \vdots \\
            0& \ddots &\ddots &\ddots& 0\\
            \vdots & \ddots& B & T & B\\
            0 & \dots & 0 & B& T
        \end{array}
    \right]
\end{align*}
com :
\begin{align*}
    T =  \left[
        \begin{array}{ccccc}
            20 &-4& 0 & \dots & 0 \\
            -4& 20 & -4 & \ddots & \vdots \\
            0& \ddots &\ddots &\ddots& 0\\
            \vdots & \ddots& -4 & 20 & -4\\
            0 & \dots & 0 & -4& 20
        \end{array}
    \right], \text{ e }
    B = 
    \left[
        \begin{array}{ccccc}
            -4 &-1& 0 & \dots & 0 \\
            -1& -4 & -1 & \ddots & \vdots \\
            0& \ddots &\ddots &\ddots& 0\\
            \vdots & \ddots& -1 & -4 & -1\\
            0 & \dots & 0 & -1& -4
        \end{array}
    \right]
\end{align*}

A maioria das linhas são diagonalmente dominantes (não estritamente):
\begin{align*}
    20 = 4 +4 + 2(1+4+1)
\end{align*}
mas elementos de cantos são estritos, por exemplo da primeira linha:
\begin{align*}
    20 > 4 + 1 + 4 +1
\end{align*}
Somando isso, ao fato dos termos da diagonal principal serem positivos e todos os outros são negativos ou zero, 
temos que $L_h$ é irredutivelmente diagonalmente dominante, portanto possui inversa e o esquema é estável.


\section{Resultados numéricos}

Para este trabalho o sistema $L_h u_h = q_h$ foi resolvido com o método 
iterativo de Gauss-Seidel. Abaixo podemos ver alguns exemplos gerados a partir do código em anexo.



\section{Interpretação dos resultados e conclusões}

Podemos ver que a ordem $O(h^4)$ produz resultados altamente precisos


% \bibliographystyle{plain} 
% \bibliography{refs}
\end{document}